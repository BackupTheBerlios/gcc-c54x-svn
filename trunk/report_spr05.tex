% Report for ECS 199, Spring 05
% Bryan Richter
% File created 15 June 2005

\documentclass{article}
\usepackage[margin=1in]{geometry}
\author{Bryan Richter\\ECS 199, Spring Quarter '05\\UC Davis}
\title{Porting GCC, Part II}

%use \file{blah} to list a file named blah.
\newcommand{\file}[1]{{\tt#1}}

\begin{document}

\maketitle

This report is a follow-up to an earlier report that documents the work
undertaken by the author to port GCC to a new target. For information regarding
rationale, GCC, and the target architecture TMS320C54\emph{x}, see the previous
report.

This report will be significantly slimmer than the previous, in part because I
will not repeat the background information, but also in part because I was no
longer the significant contributor to the project's code. As I will discuss,
this port became an international collaboration in the spirit of Open Source
Software.

\section{Work Done}

After spending three months on my own, reading documentation and writing macros
in the header file, I realized it was time to rejoin the GCC community. I began
by contacting the organizers of previous port efforts and asking their
permission to go ahead with my own (rather than add my contributions to their
projects). I did this for a three reasons: they other projects showed no
activity, I wanted to use the Subversion system rather than CVS, and I figured I
would have more free time than some of the other projects leaders, which would
make me a better candidate for project maintenance.

After receiving the go-ahead, my next step was to create a project site. I chose
BerliOS\footnote{\tt http://www.berlios.de} as a host site, and created a rough
website for the project\footnote{\tt http://gcc-c54x.berlios.de}. Also, I
immediately was contacted by some interested developers. I had barely made my
presence known before beginning to collaborate with people from around the globe.

One person in particular, Jonathan Bastien-Filiatrault of Canada, had
independently done much of the same work I had. We consolidated our code
(choosing my own where they overlapped, since we determined mine to be more
organized), and Jonathan immediately began to add code at a phenomenal rate. At
this point, I relegated myself to the role of `interested onlooker,' as
Jonathan's coding speed and free time surpassed my own.

\section{Results}

My personal accomplishment this quarter was to kick off an Open Source Project
with global contributors. The overall project's accomplishments, due mostly to
Mr. Bastien-Filiatrault's efforts, have been quite satisfactory. For one, we can
now compile GCC. This allows us to check our work by seeing how it affects test
compilations, which is enormously useful. I had to do all my work blind last
quarter, as I could get no feedback on my macros. Further, important aspects of
the compiler actually \emph{work}, and it outputs correct(?) assembly for
argument-passing and simple arithmetic instructions. There is also a much larger
framework in place to add on more and more instructions.

\section{Future Work}

There is, of course, much yet to be done. Jonathan has expressed the feeling
that his limits of GCC understanding have been reached, and his incredibly fast
code production has slowed. This means that there is plenty of work left to do
in the port itself, from optimizing macros, to adding more instructions and zany
RTXs, to cleaning up the files and preserving order. Outside of the port, more
work could be put into making the webpage more attractive and informational.
Another potential project, which would benefit a large number of people, would
be a test-harness that could be used to verify the correctness of macros or sets
of macros. I am unaware of any existing test platform, and I am not clear on its
feasibility, but I can see its benefit. 

More in line with this particular project's goals and motivation, other future
work could include merging the work into GCC. While this is not a difficult task
in itself, the port must be taken to a respectable state of working order before
its inclusion is merited. Other possibilities would be to obtain or create a
library that would allow GCC to compile just not for the TMS320C54\emph{x}, but
for the Neuros in particular. Along this vein, creating the necessary additions
to the port such that it truly deserves the \emph{x} on TMS320C54\emph{x}
(instead of working solely for the 'C5416) would be a good project. Finally,
figuring out and documenting the steps necessary to build up to compiling the
Neuros' firmware would be a welcome addition to the OSS community.

\begin{thebibliography}{99}
\bibitem{gccint} \emph{GCC Internals Manual}.
\texttt{http://gcc.gnu.org/onlinedocs/gccint/}, Mar. 20 2005. Free Software
Foundation, Inc.

\bibitem{refset1} \emph{TMS320C54x DSP Reference Set, Volume 1}. Texas
Instruments, Inc. Fort Worth, Texas, 2001.

\bibitem{refset2} \emph{TMS320C54x DSP Reference Set, Volume 2}. Texas
Instruments, Inc. Fort Worth, Texas, 2001.

\end{thebibliography}
\end{document}
